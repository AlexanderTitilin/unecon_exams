\documentclass{scrartcl}
\usepackage[utf8]{inputenc}
\usepackage[T2A]{fontenc}
\usepackage[russian]{babel}
\usepackage{amssymb}
\usepackage{amsmath}
\usepackage{amsthm}
\usepackage{listings}
\newtheorem{theorem}{Теорема}
\newtheorem{definition}{Определение}
\newtheorem{corollary}{Следствие}[theorem]
\newtheorem{lemma}[theorem]{Лемма}
\usepackage{graphicx}
\graphicspath{ {./images/} }
\title{Экзамен по алгебре и геометрии}
\author{Титилин Александр}
\date{}
\begin{document}
    \maketitle
    \paragraph{Ананлитическая Геометрия}
    \section{Геометрические векторы.}
    \begin{definition}
        Вектор -- направленный отрзок, который характеризуется длиной и направлением.$\overline{AB}$ -- вектор, А -- начало (точка приложения),
        B -- конец.
        $|AB|$ -- Длина вектра
    \end{definition}
    \begin{definition}
       Векторы коллинеарны ($\overline{a} \parallel \overline{b}$) , если $L$ -- прямая,  $\overline{a} \parallel L \land \overline{b} \parallel L$
    \end{definition}
    \begin{definition}
        $a_0$ орт вектора $\overline{a}$ , если $\overline{a_0}$ сонаправлен $\overline{a} \land |\overline{a_0} = 1$
    \end{definition}
    \begin{definition}
        $\overline{a} = \overline{b}$ := $|a| = |b| \land a\parallel b \land$ а сонаправлен b.
    \end{definition}
    \begin{definition}
        Суммой $\overline{a}$ и $\overline{b}$ называют вектор, идущий из начала $\overline{a}$ в конец $\overline{b}$, если $b$ приложен к концу  $\overline{a}$
    \end{definition}
    \includegraphics{img/vsum.pdf}
    \begin{definition}
        Произведение вектора на число -- вектор, коллинеарный исходногому и имеющий его длину умноженную на число
    \end{definition}
    \begin{theorem}
       \begin{enumerate}
           \item $\overline{a} + \overline{b} = \overline{b} + \overline{a}$
                \item
                    $(\overline{a} + \overline{b}) + \overline{c} = \overline{a} + (\overline{b} + \overline{c})$
                 \item
                     $\exists!  \overline{0}: \overline{a} + \overline{0} = \overline{a}$
                 \item $\forall \overline{a} \exists  -\overline{a} : \overline{a}
                     + (-\overline{a}) = \overline{0}$
                \item $\alpha(\beta \overline{a}) = ( \alpha\beta ) \overline{a}$
                \item $( \alpha + \beta ) \overline{a} = \alpha \overline{a} + \beta \overline{a}$
       \end{enumerate} 
    \end{theorem}
    \section{Линейная зависимость векторов.}
    \begin{definition}
        Набор векторов $a_1,a_2,a_3,\dots,a_{n}$ называется линейно зависимым, если существует набор чисел, где хоть одно не равно 0 и $\alpha_1 a_1 + \alpha_2 a_2 + \dots \alpha_{n}a_{n} = 0 $
    \end{definition}
    \begin{theorem}
         \begin{enumerate}
             \item 
        $\overline{0}$ -линейно зависимый
    \item $a_1,\dots,0,\dots,a_n$ - линейно зависимый
        \item
            $a_{1}, \dots,a_{i}$ - линейно зависимый,$a_1,\dots a_{i},\dots a_{n}$ -- линейно зависимый.
        \item
            $a_1, \dots, a_{i},\dots,a_{n}$ - линейно зависимый $\iff$  $a_{i} =
            \sum_{j \neq i} a_{i}$
         \end{enumerate}
    \end{theorem}
    \begin{proof}
        \begin{enumerate}
            \item $1 * \overline{0} = \overline{0}$ 
                \item
                    $0a_1 + \dots + 1 * \overline{0} + \dots + 0 * \overline{a_{n}}= 0$
                    \item
                        $\exists {\alpha_1 \dots \alpha_{i}} \sum \alpha_{j}a =0 ~ 
                        \alpha_1 a_1 + \dots \alpha_{i}a_{i} + \dots 0*a_{n} = 0$
        \end{enumerate}
    \end{proof}
    \section{Линейная зависимость трех и четырех векторов в плоскости и пространстве.}
    \begin{theorem}
        $\overline{a}, \overline{b}$ линейно зависимые $\iff$  $\overline{a} \parallel \overline{b}$
    \end{theorem}
    \begin{proof}
        $\rightarrow$  $\alpha \overline{a} + \beta \overline{b} = \overline{0}$.
        Пусть $\beta \neq 0 \implies \overline{b} = (-\frac{\alpha}{\beta}) \overline{a} \implies a \parallel b$ \\
        $\leftarrow$  $\overline{b} = \lambda \overline{a} \implies \overline{b} - \lambda \overline{a} = \overline{0}$
    \end{proof}
    \begin{theorem}
        $\overline{a},\overline{b},\overline{c}$ линейно зависимые $\iff$  $\overline{a}, \overline{b},\overline{c} \parallel \Pi$
    \end{theorem}
    \begin{proof}
        $\rightarrow$.  $\alpha \overline{a} + \beta \overline{b} + \gamma \overline{c} \gamma \neq 0 \implies \overline{c} = -\frac{\alpha}{\gamma}\overline{a} - \frac{\beta}{\gamma} \overline{b}$. Векторы лежат в одной плоскости.\\
        $\rightarrow$. Если одна из пар коллинеарна, то она линейно зависимая, и мы можем просто оставшийся вектор на 0 умножить. Иначе переносим все прямые на одну плоскоскость в одну точку. Через конец вектора с проводим прямые паралельные  a и b.
        Эти прямые пересекают  $\nu \overline{a}$ и $\mu \overline{b}$. $\overline{ c} = \nu \overline{a} + \mu \overline{b}$
    \end{proof}
    \begin{theorem}
        Любые четыре вектора линейно зависимы.
    \end{theorem}
    \begin{proof}
        Если 3 вектора компланарны, то все понятно.
    \end{proof}
    \section{Проекция вектора на ось.}
    \begin{definition}
        u -- ось (направленная прямая), $\vec{a} = \vec{AB}$, $A' B'$ основания перпендикуляров, опущенных на  $A, B$.  $\pm \mid \vec{A'B'} \mid$ -- проекция $\vec{a}$ на u.
        Знак зависит от направления вектора. Если он сонаправлен с осью, то плюс иначе минус.
    \end{definition}
    \begin{theorem}
        Проекция вектора $\vec{a}$ на ось u равна $|\vec{a}| * \cos{\phi}$, где $\phi$  угол наклона вектора к оси.
    \end{theorem}
    \begin{theorem}
        $Pr_u( \vec{a} + \vec{b} ) = Pr_{u}\vec{a} + Pr_{u}\vec{b}$\\
        $Pr_{u}(\alpha \vec{a}) = \alpha Pr_{u} \vec{a}$
    \end{theorem}
    \section{Базис аффиные координаты}
    \begin{definition}[базис в пространстве]
        $\overline{a},\overline{b},\overline{c}$ линейно независимы. Они образует базис если $ \forall \overline{d} \exists  \lambda ,\mu, \nu: \overline{d} = \lambda \overline{a} + \mu \overline{b} + \nu \overline{c}$
    \end{definition}
    \begin{definition}[базис на плоскости]
        $\vec{a}, \vec{b} $ линейно независимы, они образуют базис если $\forall  \vec{c} : \exists \lambda, \nu ~ \vec{c} = \lambda \vec{a} + \nu \vec{b}$
    \end{definition}
    \begin{theorem}
        Любые три линейно-независимые вектора однозначно раскладываются по базису относительно четвертого.
    \end{theorem}
    \begin{proof}
        $\vec{d} = \lambda \vec{a} + \nu \vec{b} + \mu \vec{c} \land \vec{d} = \alpha \vec{a} + \beta \vec{b} + \gamma \vec{c}  (\lambda - \alpha)\vec{a} + (\nu - \beta) \vec{b} + (\mu - \gamma) \vec{c} = \vec{0} $. Векторы линейно независимые $\lambda = \alpha , \mu =\beta , \nu = \gamma$
    \end{proof}
    \begin{theorem}
        При сложение двух векторов их координаты относительно одного базиса складываются. При умножении на число умножаются на это число.
    \end{theorem}
    \begin{proof}
        \[
        \vec{d_1} = \alpha_1 \vec{a} + \beta_1 \vec{b} + \gamma_1 \vec{c}
        .\] 
        \[
        \vec{d_2} = \alpha_2 \vec{a} + \beta_2 \vec{b} + \gamma \vec{c}
        .\] 
        \[
        \vec{d_1} + \vec{d_2} = (\alpha_1 + \alpha_2) \vec{a} + (\beta_1 + \beta_2)\vec{b} + (\gamma_1 + \gamma_2) \vec{c}
        .\] 
        \[
        \lambda \vec{d_1} = (\lambda \alpha_1)\vec{a} + (\lambda \beta_1)\vec{b} + (\lambda  \gamma_1) \vec{c}
        .\] 
        \begin{definition}
            Аффиные координаты задаются базисом $\vec{a}, \vec{b},\vec{c}$ 
            и точкой O (началом координат)
        \end{definition}
    \end{proof}
    \begin{definition}[Декартовы координаты]
        $\vec{i},\vec{j},\vec{k}$ -- взаимно ортогональные векторы длины 1. Разложение по такому базису -- декартовы координаты
    \end{definition}
    \begin{theorem}
        Декартовы координаты вектора $\vec{d}$ равны его проекциям на оси $Ox Oy Oz$
    \end{theorem}
    \begin{theorem}
        $\alpha,\beta,,\gamma$ углы наклона вектора  $\vec{d}$  к осям $Ox,Oy,Oz$
         \[
             \cos^2{\alpha} + \cos^2{\beta} + \cos^2{\gamma} = 1
        .\] 
    \end{theorem}
    \begin{proof}
        \[
            \vec{d} = \{X,Y,Z \}
        .\] 
        \[
            X = |\vec{d}| \cos{\alpha}
        .\] 
        \[
            \cos{\alpha} = \frac{X}{\sqrt{X^2 + Y^2 + Z^2}}
        .\] 
        \[
            \cos{\beta} = \frac{Y}{\sqrt{{X^2 + Y ^2 + Z ^2}} }
        .\] 
        \[
            \cos{\gamma} = \frac{Z}{\sqrt{X^2 + Y^2 + Z ^2} }
        .\] 
        Каждую из этих хреней в квадрат возводим и складываем теорема доказана.
    \end{proof}
    \begin{definition}
        M делит отрезок $M_1M_2$ в отношении $\lambda$ , если  $\vec{M_1M} = \lambda \vec{MM_2}$
    \end{definition}
    \[
    M = (x,y,z)
    .\] 
    \[
    M_1 = (x_1,y_1,z_1)
    .\] 
    \[
    M_2 = (x_2,y_2,z_2)
    .\] 
    \[
        \{x_1 - x, y_1 - y,z_1 - z\} = \lambda \{x_2 - x,y_2 - z,z_2 - z\}
    .\] 
    \section{Скалярное произведение}
    \begin{definition}
        $(\vec{a},\vec{b}) = |\vec{a}| * |\vec{b}|\cos{(\vec{a},\vec{b})}$ -- скалярное произведение векторов $\vec{a}, \vec{b}$. Так же выражается через проекции.
        \[
            |\vec{b}| \cos{(\vec{a},\vec{b})} = Pr_{a} \vec{b} \implies (\vec{a},\vec{b}) = |\vec{a}| Pr_{a}\vec{b}
        .\] 
        \[
            |\vec{a}| \cos{(\vec{a},\vec{b})} = Pr_{b}\vec{a} \implies (\vec{a},\vec{b}) = |\vec{b}| Pr_{b} \vec{a}
        .\] 
    \end{definition}
    \begin{theorem}
        $\vec{a} \perp \vec{b} \iff (\vec{a}, \vec{b}) = 0$
    \end{theorem}
    \begin{proof}
        $\rightarrow$.  $\cos{(\vec{a},\vec{b})} = 0 \implies (\vec{a},\vec{b}) = 0$\\
        $\leftarrow$.  $(\vec{a},\vec{b}) = 0 \implies |a| = 0 \lor |b| = 0 \lor \cos{(\vec{a},\vec{b})} = 0$.
    \end{proof}
    \begin{theorem}
        Если угол между векторами меньше $\frac{\pi}{2}$ cкалярное произведение больше нуля. Иначе оно меньше нуля.
    \end{theorem}
    \subsection{Алгебраические свойства.}
    \begin{theorem}
        \begin{enumerate}
            \item $(\vec{a},\vec{b}) = (\vec{b},\vec{a})$
            \item $(\vec{a} + \vec{b},\vec{c}) =(\vec{a},\vec{c}) + (\vec{b}, \vec{c})$
            \item $(\alpha \vec{a},\vec{b}) = \alpha (\vec{a},\vec{b})$
            \item $(\vec{a},\vec{a}) > 0$ если $a \neq 0$ , $(\vec{a},\vec{a})=0$, если $\vec{a} = 0$
        \end{enumerate}
    \end{theorem}
    \begin{proof}
        \begin{enumerate}
            \item очев
            \item $(\vec{a} + \vec{b},\vec{c}) = |c| Pr_{c}(\vec{a} + \vec{b}) =
                |c|Pr_{c}\vec{a} + |c| Pr_{c}\vec{b} = (\vec{a},\vec{c}) + (\vec{b},\vec{c})$
            \item $(\alpha \vec{a},\vec{b}) = |b|Pr_{b}(\alpha a) = \alpha |b| Pr_{\vec{b}}(\vec{a}) = \alpha (\vec{a},\vec{b})$
            \item $(\vec{a},\vec{a}) = |a|^2$
        \end{enumerate}
    \end{proof}
    \subsection{Скалярное произведение в координатах}
    \begin{theorem}
        \[
            \vec{a} = \{x_1,y_1,z_1\}
        .\] 
        \[
            \vec{b} = \{x_2,y_2,z_2\}
        .\] 
        \[
            (\vec{a},\vec{b}) = x_1x_2 + y_1y_2 + z_1z_2
        .\] 
    \end{theorem}
    \begin{proof}
        \[
            \vec{a} = x_1 \vec{i} + y_1 \vec{j} + z_1 \vec{k}
        .\] 
        \[
        \vec{b} = x_2 \vec{i} + y_2 \vec{j} + z_2 \vec{k}
        .\] 
        У одинаковых ортов скалярное произведение равно 1, у разных 0.
    \end{proof}
    \subsection{Угол между векторами}
    \[
    \cos(\vec{a},\vec{b}) = \frac{(\vec{a},\vec{b})}{|\vec{a}||\vec{b}|} =
    \frac{x_1x_2 + y_1 y_2 + z_1 z_2}{\sqrt{x_1^2 + y_1^2 + z_1 ^2} \sqrt{x_2^2 + y_2^2 + z_2^2}}
    .\] 
    \section{Ортогональость векторов}
    \[
    \vec{a} \perp \vec{b} \iff (\vec{a},\vec{b}) = 0 \implies x_1x_2 +y_1y_2 +z_1z_2 = 0
    .\] 
    \section{Векторное произведение}
    Назовем упорядоченную тройку векторов $\vec{a},\vec{b},\vec{c}$ правой, если
    c конца $\vec{c}$ кратчайщий поворот от $\vec{a}$ к $\vec{b}$ виден наблюдателю против часовой стрелки
    \begin{definition}[Векторное произведение]
        $\vec{c} = [\vec{a},\vec{b}]$ является векторным произведением $\vec{a},\vec{b}$, если
        \begin{enumerate}
            \item $|\vec{c}| = |\vec{a}||\vec{b}| \sin{(\vec{a},\vec{b})}$
            \item $c \perp a, c \perp b$
            \item  $\vec{a} ,\vec{b} ,\vec{c}$ правая тройка
        \end{enumerate}
    \end{definition}
    \subsection{Геометрические свойства}
    \begin{theorem}
        \begin{enumerate}
            \item $\vec{a} \parallel \vec{b} \iff [\vec{a},\vec{b}] = 0$
            \item $[\vec{a}\vec{b}]$ равен площади паралеллограмма, со сторонами $a,b$
        \end{enumerate}
        \begin{proof}
            \begin{enumerate}
                \item $\rightarrow$  $\sin{( \vec{a},\vec{b} )} = 0$\\
                    $\leftarrow$  $|a| = 0 \lor |b| = 0 \lor \sin{(\vec{a},\vec{b})} = 0$
                    \item
                        Формула из школы
            \end{enumerate}
        \end{proof}
    \end{theorem}
    \subsection{Алгебраические свойства}
    \begin{theorem}
        \begin{enumerate}
            \item $[\vec{b},\vec{a}] = - [\vec{a},\vec{b}]$
            \item $[\alpha \vec{a},\vec{b}] = \alpha [\vec{a},\vec{b}]$
            \item $[\vec{a} + \vec{b},\vec{c}] = [\vec{a},\vec{c}] + [\vec{b},\vec{c}]$
            \item $ \forall  \vec{a}, [\vec{a},\vec{a}] = 0$
        \end{enumerate}
    \end{theorem}
    \begin{proof}
        
    \end{proof}
    \subsection{Векторное произведение в координатах}
    \[
    \vec{a} = x_1 \vec{i} + y_1 \vec{j} + z_1 \vec{k}
    .\] 
    \[
    \vec{b} = x_2 \vec{i} + y_2 \vec{j} + z_2 \vec{k}
    .\] 
    \[
        [\vec{a},\vec{b}] = x_1x_2 [\vec{i},\vec{i}] +
        x_1 y_2 [\vec{i},\vec{j}] +
        x_1z_2[\vec{i},\vec{k}] +
        y_1x_2[\vec{j},\vec{i}] +
        y_1y_2[j,j] +
        y_1z_2[\vec{j},\vec{k}] +
        z_1x_2[\vec{k},\vec{i}] +
        z_1y_2[\vec{k},\vec{j}] +
        z_1z_2[\vec{k},\vec{k}]
    .\] 
    \[
    = (y_1z_2 - z_1y_2)\vec{i} - (x_1z_2-z_1x_2)\vec{j} + (x_1y_2 - y_1x_2)\vec{k} =
    .\] 
    \[
    \begin{vmatrix}
        i & j & k\\
        x_1 & y_1 & z_1\\
    x_2 & y_2 & z_2\\
\end{vmatrix}
    .\] 
    \subsection{Условие коллинеарности векторов}
    $a \parallel \vec{b} \iff [\vec{a},\vec{b}] = 0 \implies $
    \[
    y_1  z_2 - z_1 y_2 = 0
    .\] 
    \[
    x_1z_2 - z_1x_2 = 0
    .\] 
    \[
    x_1y_2 - y_1x_2 = 0
    .\] 
    \section{Смешанное произведение векторов}
    \begin{definition}[Смешаное произведние векторов]
        \[
            ([\vec{a},\vec{b}],\vec{c})
        .\] 
    \end{definition}
    \subsection{Геометрические свойства}
    \begin{theorem}
        \begin{enumerate}
            \item  $\vec{a},\vec{b},\vec{c} \in \Pi \iff ([\vec{a},\vec{b}],\vec{c}) = 0$
            \item $([\vec{a},\vec{b}],\vec{c}) = \pm V$
        \end{enumerate}
    \end{theorem}
    \subsection{Алгебраические свойства}
    \begin{theorem}
        \begin{enumerate}
            \item $([\vec{a},\vec{b}],\vec{c}) = (\vec{a},[\vec{b},\vec{c}]) = \vec{a}\vec{b}\vec{c}$
                \item $\vec{a}\vec{b}\vec{c} = \vec{b}\vec{c}\vec{a} = \vec{c}\vec{a}\vec{b} = -\vec{b}\vec{a}\vec{c} = -\vec{a}\vec{c}\vec{b} = -\vec{c}\vec{b}\vec{a}$
        \end{enumerate}
    \end{theorem}
    \subsection{Смешанное произведение через координаты.}
    \[
    \vec{a}\vec{b}\vec{c} = x_1(y_2z_3 - z_2y_3) - y_1(x_2z_3 -z_2x_3) + z_1(x_2y_3-y_2x_3) = 
    \begin{vmatrix} 
        x_1 & y_1 & z_1 \\
        x_2 & y_2 & z_2 \\
        x_3 & y_3 & z_3\\
    \end{vmatrix} 
    .\] 
    \section{Прямая линия на плоскоскости}
    \subsection{Общее уравнение}
    \[
    Ax + By + C = 0
    .\] 
    A и B не равны 0 одновременно.
    \subsection{Неполные уравнеия}
    \begin{enumerate}
        \item $Ax + By = 0$ прямая проходит через начало координат
        \item  $Ax + C = 0$ прямая паралелльная оси  $Oy$
        \item  $By + C = 0$ прямая паралелльная оси  $Ox$
        \item  $Ax = 0$ ось  $Oy$
        \item  $By = 0$ ось  $Ox$
    \end{enumerate}
    \subsection{Уравнение в отрезках}
         Если $C \neq 0 \frac{A}{-C}x + \frac{B}{-C}y = 1$ если $A \neq 0 B\neq 0  \frac{x}{a} + \frac{y}{b} = 1, a = -\frac{C}{A} B = -\frac{C}{B}$ 
    \subsection{Уравнение с угловым коэфициентом}
    Если $B \neq 0 y = kx + b, k = -\frac{A}{b}, b  = -\frac{C}{B}$
    \subsection{Канончиеское уравнение прямой.}
    \begin{definition}
        Направляющий вектор -- ненулевой вектор параллельный данный прямой.
    \end{definition}
    Надо найти уравнение прямой, проходящкй через $M_1(x_1,y_1)$ и имеющей заданный направляющий вектор $\vec{q} = \{l,m\}$. $M(x,y)$ лежит на прямой только тогда когда  $\vec{M_1M} \parallel \vec{q}$
    \[
    \frac{x - x_1}{l} = \frac{y  - y_1}{m}
    .\] 
    \subsection{Параметрические уравнения}
    \[
    \begin{cases}
        x = x_1 + lt\\
        y = y_1 + mt
    \end{cases}
    .\] 
    \subsection{Векторное уравнение}
    \[
    \vec{r} = \vec{r_0} +t \vec{1}
    .\] 
    \section{Нормальное уравнение прямой.}
    У нас есть прямая L. Прямая n проходит через начало координат и перпендикулярна n. 
    P точка их пересечения. $\vec{n}$ единичный вектор выбранный на OP
    \[
        \vec{n} = \{\cos{\theta}, \sin{\theta}\}
    .\] 
    \[
        x \cos{\theta} + y \sin{\theta}  - o
    .\] 
    \section{Уравнение плоскости}
    \paragraph{Комплексные числа и многочлены}
    \section{Комплексные числа}
    \begin{definition}
        Комплексное число -- пара действительных чисел с заданными операциями сложения и умножения
        \[
        \alpha = (a,b)
        .\] 
        \[
        \beta = c,d
        .\] 
        \[
        \alpha + \beta = (a + c, b + d)
        .\] 
        \[
        \alpha*\beta = (ac - bd,ad + bc)
        .\] 
    \end{definition}
    \subsection{Свойства сложения}
    \begin{enumerate}
        \item $\alpha + \beta = \beta + \alpha$
        \item  $(\alpha + \beta) + \gamma = \alpha + (\beta + \gamma)$
        \item $\exists ! \mathbf{0} : \forall  \alpha : \alpha + \mathbf{0} = \alpha, \mathbf{0} = (0,0)$
        \item $\forall  \alpha \exists ! -\alpha : \alpha + (-\alpha) = \mathbf{0}$ $-\alpha = (-a , -b)$
    \end{enumerate}
    \subsection{Свойства умножения}
    \begin{enumerate}
        \item $\alpha * \beta = \beta * \alpha$
        \item  $(\alpha *\beta) * \gamma = \alpha * (\beta * \gamma)$
        \item  $\exists! \mathbf{1} \forall \alpha \alpha * \mathbf{1} = \alpha, \mathbf{1} = (1,0)$
        \item $(\alpha + \beta)*\gamma = \alpha * \gamma + \beta*\gamma$
        \item  $\forall  \alpha \neq 0 \exists  ~ \alpha^{-1} : \alpha * \alpha^{-1}=  1, \alpha^{-1} = (\frac{a}{a^2 + b^2}, -\frac{b}{a^2 + b^2})$
    \end{enumerate}
    \subsection{Мнимая единица}
    \[
    i = (0,1)
    .\] 
    \[
    i^2 = (-1, 0)
    .\] 
    \[
    i^2 + (1, 0) = 0
    .\] 
    \subsection{Алгебраическая форма записи}
    \[
    \alpha = (a,b) = (a,0) + (0,b) \implies \alpha = a + bi
    .\] 
    \[
    \alpha  \pm \beta = (a \pm c) + i(b \pm d)
    .\] 
    \[
    \alpha * \beta = (ac - bd) + i(ad + bc)
    .\] 
    \subsection{Тригонометрическая форма записи комплексных чисел.}
    Комплексное число имеет интерапретацию ввиде точки на декартовых координатах, с осями $Re, Im$,  $r = |\alpha| = \sqrt{a^2 + b^2} $ -- расстояние до точки, $\theta$ -- угол между вещественной осью и радиус вектором из начала координат в точку.
    Таким образом комплексное число  $\alpha = a+bi = r(\cos{\theta} + i\sin{\theta})$
     \[
         z_1 * z_2 = r_1(\cos{\alpha} + i\sin{\alpha}) * r_2(\cos{\beta} + i\sin{\beta}) + r_1r_2 (\cos{( \alpha + \beta )} + i \sin{( \alpha + \beta )})
    .\] 
    \[
        \frac{z_1}{z_2} = \frac{r_1}{r_2} (\cos{( \alpha - \beta )} + i \sin{( \alpha - \beta )})
    .\]
    \section{Возведение в степень.}
    \subsection{В алгебраической форме.}
    \[
    z = a + ib
    .\] 
    \[
        z^{n} = (a + ib)^{n} = \sum_{i=0}^{n} \binom{n}{i} a^{n - i}(bi)^{i}
    .\] 
    \[
    i^{4k} = 1
    .\] 
    \[
    i^{4k + 1} = i
    .\] 
    \[
        i^{4k + 2} = -1
    .\] 
    \[
        i^{4k + 3} = -i
    .\] 
    \subsection{В тригонометрической форме.}
    \begin{theorem}
        \[
            \alpha = r (\cos{\theta} + i \sin{\theta})
        .\] 
        \[
            a^{n} = r^{n} (\cos{n\theta} + \sin{n\theta})
        .\] 
    \end{theorem}
    \begin{proof}
        По индукции. База очевидна
        \[
            a * a^{n} = r^{n} (\cos{n\phi} + i\sin{n\phi}) r (\cos{\phi} + i\sin{\phi}) = r^{n + 1} (\cos{((n + 1) \phi)} + i \sin{((n + 1) \phi)})
        .\] 
    \end{proof}
    \section{Извлечение корня в тригонометрической форме}
    \[
        \sqrt[n]{r( \cos{\alpha} + i\sin{\alpha} )}  =
        \sqrt[n]{r}(\cos{\frac{\alpha + 2\pi k}{2}} + i\sin{\frac{\alpha + 2\pi k}{n}}), k = 0\dots n - 1
    .\] 
\end{document}

